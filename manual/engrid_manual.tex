\documentclass[10pt,a4paper,british]{book}
\usepackage[T1]{fontenc}
\usepackage[utf8]{inputenc}
\usepackage{array}
\usepackage{color}
\usepackage{graphicx}

\makeatletter

%%%%%%%%%%%%%%%%%%%%%%%%%%%%%% LyX specific LaTeX commands.
%% Because html converters don't know tabularnewline
\providecommand{\tabularnewline}{\\}

\usepackage{amsmath}
\usepackage{amsfonts}
\usepackage{graphicx} 
\usepackage{amssymb}
\usepackage{calc,layouts}
\usepackage{titlesec}
\usepackage{fancyhdr}
\usepackage{xspace}
\usepackage{color}
\usepackage{ifthen}
\usepackage{boxedminipage}
\usepackage{lastpage}
\usepackage{setspace}
\usepackage{float}
\usepackage{afterpage}
\usepackage[german,british]{babel}

\setlength\textheight{240mm}
\setlength\textwidth{150mm}
\setlength\evensidemargin{15mm}
\setlength\oddsidemargin{15mm}
\setlength\topmargin{-15mm}
\setlength\headheight{15mm}
\setlength\parindent{0mm}
\setlength\parskip{2mm}

\usepackage[T1]{fontenc}
\usepackage[scaled]{helvet}
\renewcommand*\familydefault{\sfdefault}
\linespread{1.3}\selectfont

\titleformat{\chapter}{\LARGE \bf}{\hspace{-20mm}\parbox[b]{20mm}{\thechapter}}{0mm}{}
\titleformat{\section}{\Large \bf}{\hspace{-20mm}\parbox[b]{20mm}{\thesection}}{0mm}{}
\titleformat{\subsection}{\large \bf}{\hspace{-20mm}\parbox[b]{20mm}{\thesubsection}}{0mm}{}
\titleformat{\subsubsection}{\bf}{\hspace{-20mm}\parbox[b]{20mm}{\thesubsubsection}}{0mm}{}

\newcommand\engitsversion{draft}
\newcommand\engitsauthor{Author}
\newcommand\engitstitle{Title}
\newcommand\engitsdate{\today}
\newcommand\engitscopyright{Copyright by enGits GmbH}
\newcommand\setengitsversion[1]{\renewcommand\engitsversion{#1}}
\newcommand\setengitstitle[1]{\renewcommand\engitstitle{#1}}
\newcommand\setengitsauthor[1]{\renewcommand\engitsauthor{#1}}
\newcommand\setengitsdate[1]{\renewcommand\engitsdate{#1}}
\newcommand\setengitscopyright[1]{\renewcommand\engitscopyright{#1}}
\newcommand\eg{ENGRID\ }
\newcommand\foam{OpenFOAM\textsuperscript{\textregistered}\ }
\newcommand\netgen{NETGEN\ }
\newcommand\egv{1.0}
\newcommand\sqt{\char`\"{}}
\newcommand\eqt{\char`\"{}\ }
\newcommand\arr{\guillemotright\ }
\newcommand\menu[1]{\textcolor{blue}{\it \hspace{5mm} #1}}


\newcommand\makeengitstitle[1]
{
  \pagestyle{empty}
  \setengitstitle{#1}
  
  \hspace{0mm}
  \begin{minipage}[t]{130mm}
    {
      \vspace{3mm}
      \begin{center}
        \includegraphics[width=75mm]{logos/titlelogo.png}
        $~$\\
        \vspace{10mm}
        \large
        \begin{tabular}{l}
          \begin{tabular}{l}
            \vspace{-1mm} enGits GmbH\\
            \vspace{-1mm} Marie-Curie-Strae 8\\
            \vspace{-1mm} 79539 Lrrach\\
            \vspace{0mm} Germany\\
          \end{tabular}\\
          \begin{tabular}{lcl}
            \vspace{-1mm} phone & : & +49 (0)7621 5500 530 \\
            \vspace{-1mm} fax & : & +49 (0)7621 5500 539 \\
            \vspace{-1mm} email & : & info@engits.com \\
          \end{tabular}\\
        \end{tabular}
        \vspace{3mm}$~$\\
      \end{center}
    }
  \end{minipage}
  
  \begin{minipage}[c][100mm][c]{130mm}
    \begin{center}
      {
        \begin{spacing}{1.5}
          \bf 
          \huge 
          \engitstitle
        \end{spacing}
      }
    \end{center}
  \end{minipage}
  \\
  \begin{minipage}{130mm}
    \begin{center}
      \begin{tabular}{|p{3cm}|p{81mm}|}
        \hline
        title & \engitstitle \\
        \hline
        date & \engitsdate \\
        \hline
        author & \engitsauthor \\
        \hline
        document-version & \engitsversion \\
        \hline
      \end{tabular}
    \end{center}
  \end{minipage}
  \clearpage
  \begin{minipage}{130mm}
    \begin{flushleft}
      {\engitscopyright}
    \end{flushleft}
  \end{minipage}
  \clearpage
  \pagestyle{fancy}

  \pagenumbering{roman}
  \tableofcontents
  \cleardoublepage
  \pagenumbering{arabic}
}

\newenvironment{engitsstandard}{}{\setlength\parskip{2mm}}
\newenvironment{engitstoc}{\setlength\parskip{0mm}}{\setlength\parskip{2mm}}

\pagestyle{fancy}
\fancyhf{}
\fancyhfoffset{0mm}
\fancyhfoffset[LO,LE]{20mm}
\fancyhead[LE,RO]{
  \includegraphics[width=50mm]{logos/titlelogo.png}
}
\fancyhead[LO]{\rightmark}
\fancyhead[RE]{\leftmark}
\fancyfoot[LE,RO]{\thepage}

\definecolor{darkred}{rgb}{0.5,0,0}
\newcommand\important[1]
{
  $~$\\
  \begin{minipage}{15mm}
    \includegraphics[width=1cm]{figures/important}
  \end{minipage}
  \begin{minipage}{110mm}
    \begin{flushleft}
      {\textcolor{darkred}{#1}}
    \end{flushleft}
  \end{minipage}
  \\
  \vspace{3mm}
}

\makeatother

\usepackage{babel}

\usepackage{rcs}
\RCS $Revision: 1.4 $

% floating stuff ///
% 
\setcounter{topnumber}{1}
\setcounter{bottomnumber}{1}
\renewcommand{\topfraction}{0.8}
\renewcommand{\bottomfraction}{0.8}
\renewcommand{\textfraction}{0.5}
\newcommand{\flush}{\afterpage{\clearpage}}



\begin{document}


\selectlanguage{british}
\setengitsauthor{O. Gloth}
\setengitsversion{CVS-\RCSRevision}


\setengitscopyright{Copyright \copyright 2008 enGits GmbH.\\
Permission is granted to copy, distribute and/or modify this document under the terms of the GNU Free Documentation License, Version~1.2 or any later version published by the Free Software Foundation; with
no Invariant Sections, no Front-Cover Texts, and no Back-Cover Texts. A copy of the license is included in the section entitled \char`\"{}GNU~Free~Documentation~License\char`\"{}.}


\makeengitstitle{enGrid manual}

\chapter{Introduction}

\eg is an open-source mesh generation software with CFD applications in mind. \eg uses the NETGEN \cite{netgen:2008} library for tetrahedral grid generation and an in-house development for prismatic boundary layer grids. Internally, \eg uses the VTK \cite{vtk:2008} data structures as well as the {*}.vtu file format. To create grids for
Currently \eg cannot generate surface grids. In order to create a volume grid it is required to import an existing surface mesh. Gmsh \cite{gmsh:2008} is an excellent open-source tool to create surface triangulations for \eg. Gmsh is able to import STEP and IGES files and it can also be used for simple geometry modelling. 

The \egv release of \eg provides native export to OpenFOAM\textsuperscript{\textregistered}\footnote{\foam is a registered trade mark of OpenCFD\textregistered Limited}\cite{openfoam:2008}. For future releases, export capabilities for complete \foam cases (including boundary conditions) and support for polyhedral cells are planned as well.

\eg is released under the GPL and we hope that it is a useful addition to the open-source CFD community. So far the implemented algorithm proved to be quite robust and it does not require much user interaction. Figure \ref{fig:Introduction1}shows a boundary layer grid that has been created around the geometry of what could be a toy plane.

\important
{
  This manual is very much a work in progress and does
  not claim to be finished, comprehensive, complete, or anything else.
  We hope that, even in this early stage, it offers a little help while
  using \eg!
}


\begin{figure}
\begin{centering}
\includegraphics[width=7cm]{figures/DeltaWing_02}
\hspace{2mm}
\includegraphics[width=7cm]{figures/DeltaWing_01}\\
\end{centering}
\caption{Prismatic boundary layer created by \eg}
\label{fig:Introduction1}
\end{figure}

\section{Current Release (\egv)}

\begin{itemize}
\item volume grids from existing surface triangulations (no surface meshing
support yet, but planned for future releases)
\item prismatic boundary layer support
\item GUI based on Qt4
\item direct export to OpenFOAM
\item experimental support for polyhedral grids in OpenFOAM
\end{itemize}

\section{Supported Platforms}

\begin{itemize}
\item \eg is developed on a LINUX system (OpenSUSE 10.3), using Qt-4.4.1,
VTK 5.2, and an SVN snapshot of NETGEN.
\item A Windows executable for Windows-XP (32bit) is also available.
\end{itemize}

\section{Supported File Formats}

\begin{itemize}
\item VTK unstructured grids in XML format (\eg's native format)
\item VTK poly data in XML format (import)
\item legacy VTK files (import)
\item OpenFOAM (export)
\item Gmsh (import \& export)
\item STL (import \& export)
\item NETGEN neutral format (export)
\end{itemize}


\chapter{Using \eg}


\section{Compilation and Installation}


\subsection{\eg on a UNIX system}

\subsubsection{Requirements}
\eg requires Qt-4.X and VTK-5.X. We use Qt-4.4.1 and VTK-5.2 but \eg should also compile with earlier versions. Please report any problems to the mailing list. We would, however, also appreciate if you report success with other versions than the ones mentioned before.

\important{Please make sure that VTK is compiled with GUI support for Qt-4.}

\subsubsection{Compilation}
\eg uses Qt's qmake tool to provide a platform independent compilation
mechanism. The source distribution has the following structure:
\begin{itemize}
\item enGrid\_\egv
\begin{itemize}
\item math
\item netgen\_svn
\item resources
\end{itemize}
\end{itemize}
First of all, you'll need to set up the following environment variables:
\begin{itemize}
\item VTKLIBDIR : Directory containing the VTK libraries
\item VTKINCDIR : Directory containing the VTK header files
\end{itemize}

On Debian or Ubuntu:
\begin{verbatim}
  export VTKLIBDIR=/usr/lib/
  export VTKINCDIR=/usr/include/vtk-5.0/
\end{verbatim}

On OpenSUSE:
\begin{verbatim}
  export VTKLIBDIR=/usr/lib64
  export VTKINCDIR=/usr/include/vtk
\end{verbatim}

Then to compile \eg you need to first compile the NETGEN library. We have created a Qt project file and a shell script to simplify this. In the main source directory simply type \"{}./build-nglib.sh\"{}. This downloads the latest source code from NETGEN's SVN repository and compiles the necessary library. If this fails, please follow the instructions in the next paragraph.
Otherwise compile \eg with the following steps:
\begin{enumerate}
\item change into the main source directory
\item type qmake
\item type make
\end{enumerate}

\subsubsection{Compiling NETGEN release}
This paragraph is only of interest if the \"{}./build-nglib.sh\"{} command failed.
Download the latest stable release of NETGEN and place it in the \sqt netgen\_svn\eqt folder. The following steps should get you a working NETGEN library:
\begin{enumerate}
\item create a folder \sqt netgen\_svn/netgen-mesher\eqt
\item unpack the netgen-X.Y.Z.tar.gz file
\item move the folder \sqt netgen-X.Y.Z\eqt to \sqt netgen\_svn/netgen-mesher/netgen\eqt
\item change directory to \sqt netgen\_svn\eqt
\item type qmake 
\item type make
\end{enumerate}
After a successful compilation of the NETGEN library you can procede with the compilation of \eg as described in the previous paragraph.

\subsubsection{Installation}
There is no installation script yet. You can simply run \eg from the source directory by typing \sqt ./engrid\eqt or you copy the binary to a place where it will be found by the system (e.g. /usr/local/bin).

\subsection{Installing the Windows Binary}

This should, hopefully, be straightforward:
\begin{enumerate}
\item Download and save the installer 
\item Run the installer 
\item Start \eg
\end{enumerate}
Of course it is quite possible, if not even likely, that there are issues on certain systems. Please report any problems to the mailing list.

\clearpage

\section{Tutorials}

\subsection{Tutorial 1: Creating a First Mesh}

\begin{figure}
\begin{centering}
\includegraphics{figures/Throttle}
\par\end{centering}
\caption{Throttle geometry}
\label{fig:throttle1}
\end{figure}

\subsubsection{Description}

This tutorial will demonstrate how to read a surface mesh and create a volume mesh for a CFD simulation. Figure \ref{fig:throttle1}shows the geometry which will be used for this tutorial; it represents an adjustable throttle. The file containing the surface mesh for this tutorial is called \sqt Throttle.msh\eqt and it can be downloaded from the \eg download page. 

\subsubsection{Importing the Surface Mesh}
To start, please import the file choosing 

\menu{Import \arr Gmsh \arr v2.0 (ASCII)}

from the menu bar. A file-dialogue will show and you can browse for the file and open it. Figure \ref{fig:T1_scr01} shows a screen-shot of \eg after importing the file. You can use the mouse to rotate, move, and zoom the view. This mouse interaction is the default mouse interaction provided by VTK.

\eg colours the faces of the surface grid in order to determine which side of the surface is inside a flow domain and which is outside. The outside is coloured in a pale green, but figure \ref{fig:T1_scr01} shows pale yellow; this means the surface is wrongly oriented and it needs to be corrected. To do this, please choose 

\menu{Mesh \arr change surface orientation}

from the menu-bar. Afterwards the surface will be oriented correctly.


\begin{figure}
  \begin{centering}
    \includegraphics[width=14cm]{figures/tutorials/T1/scr01}
    \par
  \end{centering}
  \caption{After importing the surface mesh}
  \label{fig:T1_scr01}
\end{figure}

\subsubsection{Defining Boundary Conditions}
Unfortunately all faces belong to the same boundary condition and thus it is not possible to see inside the domain. To change this you can pick a surface on the side of the cylindrical geometry and then change its boundary condition to a different value. To pick a face, please point the mouse over a triangle and press the {}``P'' key on your keyboard. Afterwards you should see something similar to figure \ref{fig:T1_scr02}. To change the boundary code, please select 

\menu{Mesh \arr set boundary code}. 

A small dialogue will pop up and it offers to select a feature angle and a new boundary code. The new boundary code should be set to \sqt 2\eqt and the feature angle can remain at 45 degrees. With this setting you should set the whole side of the cylinder to a new boundary code and the faces should disappear, because they have not been selected for viewing yet. Now, do the same with the top (boundary condition 3) and the bottom (boundary condition 4) of the cylinder. To get rid of the red box, please point the mouse into an empty space and press {}``P'' again. Now would be a good time to save your work. Select 

\menu{File \arr Save Grid As}

to save the file. 

\begin{figure}
  \begin{centering}
    \includegraphics[width=14cm]{figures/tutorials/T1/scr02}
    \par
  \end{centering}
  \caption{After picking a face}
  \label{fig:T1_scr02}
\end{figure}

\begin{figure}
  \begin{centering}
    \includegraphics[width=14cm]{figures/tutorials/T1/scr03}
    \par
  \end{centering}
  \caption{Physical walls for prismatic boundary layer}
  \label{fig:T1_scr03}
\end{figure}

\clearpage
\subsubsection{Create Volume Mesh}
Creating a first volume mesh, including the boundary layer, is fairly easy now. First choose

\menu{View \arr boundary codes}
 
and select the boundary conditions 1 and 2, because these represent the physical walls of the geometry. You should now have something similar to figure \ref{fig:T1_scr03}. To create the grid, simply select

\menu{Mesh \arr create prismatic boundary layer}, 

select the boundary conditions 1 and 2 and click {}``OK''. You can watch the progress in the output window on the left side of the screen. This output window can be detached, moved somewhere else, or hidden completely. \eg indicates that it is busy in the status line at the bottom of the window. After \eg has finished you can select {}``tetras'' and {}``wedges'' from
the available options on the right side of \eg's main window. In order to see inside you should also enable the clipping options. The origin of the clipping plane can be set to (0,0,0) and the normal vector to (0,0,-1). If you now select to view only boundary condition 1 and choose

\menu{View \arr redraw}
 
your screen should look similar to figure \ref{fig:T1_scr04}. To get a nice tetrahedral part of the grid it is advisable to execute

\menu{Mesh \arr create improve volume mesh (NETGEN)}

once or twice. The mesh size distribution is not ideal for the first run of NETGEN. \eg uses an existing volume grid to compute a mesh size distribution and uses this as input for the next call of NETGEN. Normally you get a rather coarse tetrahedral grid together with the prismatic layer. The next call will produce a grid that might be somewhat too fine. Starting from the second call of

\menu{Mesh \arr create improve volume mesh (NETGEN)}
 
the grid should look rather nice (see figure \ref{fig:T1_scr04}).

\begin{figure}
  \begin{centering}
    \includegraphics[width=14cm]{figures/tutorials/T1/scr04}
    \par
  \end{centering}
  \caption{First volume grid}
  \label{fig:T1_scr04}
\end{figure}

\clearpage
\subsubsection{Refining the Boundary-Layer}
At the moment the boundary layer consists of a single layer of prisms. Refining the boundary layer is a straightforward process. 

\important
{
  Save the grid with the refined boundary layer to
  a different file name, or don't save it at all (just export). At the
  moment the refinement cannot be reversed and thus the grid spacing
  cannot be changed. To do this, load the file with the initial one-layer
  boundary layer and refine again.
}

To refine the boundary layer, choose

\menu{Mesh \arr divide prismatic boundary layer}.
\begin{figure}
  \begin{centering}
    \includegraphics[width=84mm]{figures/tutorials/T1/scr05}
    \par
  \end{centering}
  \caption{Parameters for boundary layer}
  \label{fig:T1_scr05}
\end{figure}

A small pop-up dialogue appears, where you can enter how many layers and how to space them. Relative spacing means, that the initial step size (on the wall) is a fraction of the average local edge length around a node. Absolute spacing uses a fixed distance for all first layer prisms; this will lead to a first layer of prisms which is parallel to the wall. Not all step sizes are possible and \eg will issue a warning if it cannot refine your boundary layer. For this tutorial the settings in figure \ref{fig:T1_scr05} should result in a decent boundary layer mesh.

\clearpage
\subsubsection{Applying a few Simple Modifications}
\eg offers the possibility to modify the grid by extruding certain boundaries. First we will add a pipe bend by applying a rotational extrusion to the upper boundary (code 3). Afterwards we will add straight sections for the in-flow and out-flow boundaries with the help of a normal extrusion. Please select

\menu{Mesh \arr extrusion}

and enter the parameters exactly as shown in figure \ref{fig:T1_scr06}. After clicking OK you should get a grid like the one shown in figure \ref{fig:T1_scr07}. The extrusion adds a new boundary code (5) which needs to be enabled with

\menu{View \arr boundary codes}

in order to view the newly created geometry. The layer height in figure \ref{fig:T1_scr06} corresponds to an angle in degrees for a rotational extrusion. To extend the pipe at the inlet a normal extrusion with the parameters from figure \ref{fig:T1_scr08} shall be used. If you do the same for boundary code 4 the grid should look like in figure \ref{fig:T1_scr09}. To simplify the later setup of a simulation it is advisable to reset the newly created boundary codes (5,6,8) to the initially set value of 2 for the pipe.

\begin{figure}
  \begin{centering}
    \includegraphics[width=132mm]{figures/tutorials/T1/scr06}
    \par
  \end{centering}
  \caption{Parameters for rotational extrusion}
  \label{fig:T1_scr06}
\end{figure}
\begin{figure}
  \begin{centering}
    \includegraphics[width=14cm]{figures/tutorials/T1/scr07}
    \par
  \end{centering}
  \caption{Grid after rotational extrusion}
  \label{fig:T1_scr07}
\end{figure}
\begin{figure}
  \begin{centering}
    \includegraphics[width=132mm]{figures/tutorials/T1/scr08}
    \par
  \end{centering}
  \caption{Parameters for normal extrusion}
  \label{fig:T1_scr08}
\end{figure}
\begin{figure}
  \begin{centering}
    \includegraphics[width=14cm]{figures/tutorials/T1/scr09}
    \par
  \end{centering}
  \caption{Grid after normal extrusion}
  \label{fig:T1_scr09}
\end{figure}
\clearpage

\subsubsection{Export the Grid to \foam}
Please select

\menu{Tools \arr edit boundary conditions}

and edit the boundary names and types according to figure \ref{fig:T1_scr10}. You can export the mesh using:

\menu{Export \arr OpenFOAM \arr  OpenFOAM}.

This will prompt you for an \foam case directory and the mesh will be directly imported to the \foam format, using the boundary names and types you have defined earlier.

\important
{
  OpenFOAM's checkMesh utility might report a bad mesh
  in case of very thin prisms. A good strategy is to export the mesh
  before the boundary layer is refined, run the checkMesh utility, and
  then - if everything looks alright - refine the boundary layer.
}
\begin{figure}
  \begin{centering}
    \includegraphics[width=74mm]{figures/tutorials/T1/scr10}
    \par
  \end{centering}
  \caption{Editing the boundary conditions}
  \label{fig:T1_scr10}
\end{figure}

\chapter{Background}

\section{Settings}

\subsection{Feature angle}
The feature angle is used to determine when an edge is a "feature edge", i.e. on the border of a "flat" surface.
A feature edge occurs when the angle between the two surface normals of a polygon sharing an edge is greater than the FeatureAngle.

Those feature edges delimit the area which will be set to the given boundary code.

\begin{figure}
  \begin{centering}
    \includegraphics[width=14cm]{figures/featureangle4}
    \par
  \end{centering}
  \caption{The feature angle}
  \label{fig:featureangle4}
\end{figure}

Consider figure \ref{fig:featureangle4}: If you picked cell 1 and chose a feature angle greater than "a", cell 1 and cell 2 will be set to the same boundary code. Otherwise not.

\section{Technical Details}

\subsection{Mesh data structures}

\subsubsection{Cells}

\subsubsection{Local cells}
Depends on:
\begin{itemize}
 \item cells
\end{itemize}

\subsubsection{Nodes}
Depends on:
\begin{itemize}
 \item cells
\end{itemize}

\subsubsection{Local nodes}
Depends on:
\begin{itemize}
 \item cells
 \item nodes
\end{itemize}

\subsubsection{N2N (node to node)}
Depends on:
\begin{itemize}
 \item cells
 \item nodes
 \item local nodes
\end{itemize}

\subsubsection{N2C (node to cell)}
Depends on:
\begin{itemize}
 \item cells
 \item nodes
 \item local nodes
\end{itemize}

\subsubsection{C2C (cell to cell)}
Depends on:
\begin{itemize}
 \item cells
\end{itemize}



\subsection{Initial Boundary Layer Generation}

\subsection{Mesh-Smoothing}

\subsection{Using NETGEN for Tetra-Meshing}


\section{Element Numbering}

\begin{figure}
  \begin{centering}
    \begin{tabular}{>{\centering}m{4cm}>{\raggedright}m{5cm}}
      \vspace{10mm}\includegraphics[width=3cm]{figures/Tetra}
      &
      \begin{tabular}{cc|cc}
        \multicolumn{2}{c|}
        {
          faces
        } 
        & 
        \multicolumn{2}{c}
        {
          edges
        }
        \tabularnewline
        \hline
        0 & 2,1,0 & 0 & 0,1 \\
        1 & 1,3,0 & 1 & 0,2 \\
        2 & 3,2,0 & 2 & 0,3 \\
        3 & 2,3,1 & 3 & 1,2 \\
          &       & 4 & 1,3 \\
          &       & 5 & 2,3 \\
      \end{tabular}
    \end{tabular}
    \par
    \end{centering}
  \caption{Tetrahedron}
  \label{fig:tetra}
\end{figure}

\begin{figure}
  \begin{centering}
    \begin{tabular}{>{\centering}m{4cm}>{\raggedright}m{5cm}}
      \vspace{10mm}\includegraphics[width=3cm]{figures/Wedge}
      &
      \begin{tabular}{cc|cc}
        \multicolumn{2}{c|}
        {
          faces
        } 
        & 
        \multicolumn{2}{c}
        {
          edges
        }
        \tabularnewline
        \hline
        0 & 0,1,2   & 0 & 0,1 \\
        1 & 3,5,4   & 1 & 0,2 \\
        2 & 3,4,1,0 & 2 & 0,3 \\
        3 & 1,4,5,2 & 3 & 1,2 \\
        4 & 0,2,5,3 & 4 & 1,4 \\
          &         & 5 & 2,5 \\
          &         & 6 & 3,4 \\
          &         & 7 & 3,5 \\
          &         & 8 & 4,5 \\
      \end{tabular}
    \end{tabular}
    \par
    \end{centering}
  \caption{Wedge/Prism}
  \label{fig:tetra}
\end{figure}

\begin{figure}
  \begin{centering}
    \begin{tabular}{>{\centering}m{4cm}>{\raggedright}m{5cm}}
      \vspace{10mm}\includegraphics[width=3cm]{figures/Hexa}
      &
      \begin{tabular}{cc|cc}
        \multicolumn{2}{c|}
        {
          faces
        } 
        & 
        \multicolumn{2}{c}
        {
          edges
        }
        \tabularnewline
        \hline
        0 & 0,3,2,1 & 0  & 0,1 \\
        1 & 4,5,6,7 & 1  & 0,3 \\
        2 & 0,1,5,4 & 2  & 0,4 \\
        3 & 3,7,6,2 & 3  & 1,2 \\
        4 & 0,4,7,3 & 4  & 1,5 \\
        5 & 1,2,6,5 & 5  & 2,3 \\
          &         & 6  & 2,6 \\
          &         & 7  & 3,7 \\
          &         & 8  & 4,5 \\
          &         & 9  & 4,7 \\
          &         & 10 & 5,6 \\
          &         & 11 & 6,7 \\
      \end{tabular}
    \end{tabular}
    \par
    \end{centering}
  \caption{Hexahedron}
  \label{fig:tetra}
\end{figure}

\section{Planned Developments}


\appendix

\chapter{GNU Free Documentation License}

\include{gpl}

\addcontentsline{toc}{section}{References}

\begin{flushleft}
\bibliographystyle{plain}
\bibliography{engits}
\par\end{flushleft}

\end{document}

